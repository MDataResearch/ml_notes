\documentclass{article}
\usepackage[utf8]{inputenc}
\usepackage{makeidx}
\usepackage{hyperref}

\title{Machine Learning Notes}
\author{Haotian Chen}
\date{}

\makeindex



\begin{document}

\maketitle

\clearpage

\tableofcontents{}

\clearpage

\section{Machine Learning Introduction}

A computer program is said to learn
from experience E with respect to some task T
and some performance measure P, if its
performance on T, as measured by P, improves
with experience E. 

\subsection{Supervised Learning}

Supervised learning algorithms build a mathematical model 
of a set of data that contains both the inputs and the desired 
outputs.

\bigskip

\noindent In supervised learning, each example is a 
pair consisting of an input object (typically a vector) 
and a desired output value (also called thesupervisory signal).

\subsubsection{Regression}

Regression analysis is a set of statistical processes 
for estimating the relationships between a dependent 
variable (often called the `outcome variable') and one 
or more independent variables (often called `predictors', 
`covariates', or `features').

\subsubsection{Classification}

Classification is the problem of identifying to which 
of a set of categories (sub-populations) a new observation 
belongs, on the basis of a training set of data containing 
observations (or instances) whose category membership is known.
       
\subsection{Unsupervised Learning}

Unsupervised learning algorithms take a set of data that 
contains only inputs, and find structure in the data, like 
grouping or clustering of data points.

\bigskip

\noindent Draw inferences from datasets consisting of input 
data without labeled responses.

\subsection{Reinforcement learning}
       
Reinforcement learning is an area of machine learning concerned 
with how software agents ought to take actions in an environment 
so as to maximize some notion of cumulative reward.

\printindex

\end{document}